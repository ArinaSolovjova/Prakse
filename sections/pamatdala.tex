% (uzņēmuma vai organizācijas apraksts; uzņēmuma vai organizācijas organizatoriskā  struktūra; pārskats par prakses gaitu; profesionālās darbības pamatuzdevumu veikšanai nepieciešamo prasmju pielietojuma prakses laikā apraksts) 

\subsection{Uzņēmuma apraksts}
Uzņēmums SIA "DIGI-INK" ir dibināts 2019.gadā. Tas pieder pie Fair Group kompānijas, kas sniedz parādu piedziņas un automatizētu dokumentu piegādes pakalpojumus Norvēģijā.
DIGI-INK atbild par klientu integrāciju un uzturēšanu, kas ietver ienākošo un izejošo dokumentu validāciju, formātu izstrādi un pielāgošanu, datu transformāciju, PDF dokumentu ģenerēšanu vai maskēšanu, dokumentu maršrutēšanu starp dažādiem piegādes kanāliem un citas saistītās darbības.
\par Uzņēmums izceļas ar strauju attīstību Norvēģijas parādu piedziņas tirgū. %TODO: update%
\par Uz šo brīdi uzņēmumā darbojas 6 darbinieki - izpilddirektors, izpilddirektora vietnieks, tehniskais direktors un pamatdarbinieki. Savstarpējai saziņai uzņēmums izmanto platformu \textit{Microsoft Teams}. Kopējam projektam tiek izmantots \textit{Gitlab}.

\subsection{Prakses gaita}
Paredzētie prakses uzdevumi ir darba autores ikdienas darba uzdevumi šajā uzņēmumā.

\subsection{Izmantotās tehnoloģijas}


% \begin{itemize}
%     \item izmantotās tehnoloģijas
%     \begin{itemize}
%         \item smooks
%         \item groovy
%         \item xsl
%         \item xml
%     \end{itemize}
%     \item kā tehnoloģijas tiek sasaistītas
%     \item demo
%     \item
% \end{itemize}
