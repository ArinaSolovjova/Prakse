
\subsection{Darbības sfēra}
Elektronisko rēķinu piegādes sistēma Norvēģija ir pieņemta jau no 2011. gada. No 2012.gada visiem piegādātājiem elektronisko rēķinu piegāde ir obligāta, ievērojot EHF formātu - PEPPOL Uzņēmējdarbības savietojamības specifikācijas (angl. Business Interoperability Specifiations) implementāciju. Uz šo brīdi ir izveidojušies vairāki standarti un veidi kā parāda informācija var tikt nogādāta nepieciešamajai personai vai uzņēmuma.
\par Lai sekmīgi nosūtītu gala klientam informāciju par parāda atmaksāšanu, ir izveidotas organizācijas, kuras darbojas kā starpnieks starp pakalpojuma devēju un pakalpojuma ņēmēju, automatizēti nogādājot nepieciešamos datus noteiktā formā.
\subsection{Uzņēmuma apraksts}
% Parādu piedziņa kam(telefons, maksas ceļi, pakalpojumi utt.)
% Backstory par Norvēģijā esošo sistēmu
Uzņēmums SIA "DIGI-INK" ir dibināts 2019.gadā. Tas pieder pie Fair Group kompānijas, kas sniedz parādu piedziņas un automatizētu dokumentu piegādes pakalpojumus Norvēģijā.
DIGI-INK atbild par klientu integrāciju un uzturēšanu, kas ietver ienākošo un izejošo dokumentu validāciju, formātu izstrādi un pielāgošanu, datu transformāciju, PDF dokumentu ģenerēšanu vai maskēšanu, dokumentu maršrutēšanu starp dažādiem piegādes kanāliem un citas saistītās darbības.
\par Uzņēmums izceļas ar strauju attīstību Norvēģijas parādu piedziņas tirgū. %TODO: update%
\par Uz šo brīdi uzņēmumā darbojas 6 darbinieki - izpilddirektors, izpilddirektora vietnieks, tehniskais direktors un pamatdarbinieki. Savstarpējai saziņai uzņēmums izmanto platformu \textit{Microsoft Teams}. Kopējam projektam tiek izmantots \textit{Gitlab}.
