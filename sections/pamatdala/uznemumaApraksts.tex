\subsection{Darbības sfēra}
\par Par dažādu pakalpojumu izmantošanu, piemēram, telekomunikāciju serviss, produktu izstrāde, maksas ceļu izmantošana, u.c., lietotājam pienākas par to samaksāt. Pakalpojuma apmaksa parasti tiek veikta pēc rēķina saņemšanas elektroniskā veidā vai fiziskā veidā saņemot to pastkastītē. Pie laicīgi neapmaksātiem rēķiniem piegādātājkompānija jeb kompānija, kas piedāvā noteikto produktu vai pakalpojumu, piegādā lietotājam atgādinājumu par rēķinu vai arī parāda izrakstu ar papildus komisijas maksām.
\par Elektronisko rēķinu piegādes sistēma Norvēģija ir pieņemta jau no 2011.gada. No 2019.gada visiem Norvēģu piegādātājiem elektronisko rēķinu piegāde ir obligāta, ievērojot EHF\footnote{Norvēģu valodā \textit{Elektronisk Handelsformat} - Elektroniskās tirdzniecības formāts} un PEPPOL BIS\footnote{Angl. \textit{Business Interoperability Specifiations} - uzņēmējdarbības savietojamības specifikācija} formātus \cite{eInvoicingNorway}. Uz šo brīdi ir izveidojušies vairāki standarti un veidi kā parāda informācija var tikt nogādāta nepieciešamajai personai vai uzņēmuma, pielietojot B2B (angl. businesss to business), B2C (angl. business to customer) vai B2G (angl. business to government) formātus.
\par Lai sekmīgi nosūtītu gala klientam informāciju par parāda atmaksāšanu, ir izveidota organizācija, kura darbojas kā starpnieks starp pakalpojuma devēju un pakalpojuma ņēmēju, automatizēti nogādājot nepieciešamos datus noteiktā formā. Pakalpojuma devējs sagatavo datus par savu klientu kāda faila formātā (\texttt{xml}, \texttt{csv} vai citā) un nosūta tos starpniekkompānijai. Tā apstrādā datus, ievērojot noteiktos standartus, sagatavo papildus nepieciešamos failus (piemēram, rēķinu \texttt{PDF} formātā) un automatizēti nosūta sagatavoto informāciju gala saņēmējam caur nepieciešamajiem piegādes kanāliem - mobilās maksāšanas aplikācijas, dažādu e-pastu servisi, printēšana piegādei uz pastkasti u.c.
% TODO: vai visas kompānijas tā dara?
\subsection{Uzņēmuma apraksts}
Uzņēmums SIA "DIGI-INK" ir dibināts 2019.gadā, Ventspilī. Tas pieder pie Fair Group kompānijas, kas sniedz parādu piedziņas un automatizētu dokumentu piegādes pakalpojumus Norvēģijā. Uzņēmums izceļas ar strauju attīstību Norvēģijas parādu piedziņas tirgū.
DIGI-INK atbild par klientu integrāciju un uzturēšanu, kas ietver ienākošo un izejošo dokumentu validāciju, formātu izstrādi un pielāgošanu, datu transformāciju, \texttt{PDF} dokumentu ģenerēšanu vai maskēšanu, dokumentu maršrutēšanu starp dažādiem piegādes kanāliem un citas saistītās darbības.
\par Uz šo brīdi uzņēmumā darbojas 6 darbinieki - izpilddirektors, izpilddirektora vietnieks, tehniskais direktors un pamatdarbinieki. Savstarpējai saziņai uzņēmums izmanto platformu \textit{Microsoft Teams}. Kopējam projektam tiek izmantota koda versijas kontroles platforma \textit{Gitlab}.

% ehfv3, efaktura, giro, aksespunkt, e2b, eboks, usdf,
% receipt