\subsection{Kas tur notiek}
\par Uzņēmumam ir izstrādāta sistēma ar dažādām klasēm, kas visvairāk atbilst saņemto datu vērtībām. Visbiežāk saņemtie dati satur tādas vērtības, kā dokumenta pamatdetaļas (izveides un izpildes datumi, dokumenta numurs), parādu uzņēmuma pamatinformācija, parādnieka pamatinformācija, parāda detaļas (summas, preces vai pakalpojuma apraksts) dokumenta kopsummas un papildinformācija.
\par Klients nosūta testa failus, kas pēc iespējas vairāk atbilst reāli izmantotajiem failiem dokumentu izplatīšanā. Saņemot klienta testa failus izstrādātājs izveido ienākošo failu kartēšanu, ielasot faila sastāvu un saglabājot noteiktajās vietās esošās vērtības atbilstošajās klasēs un klašu mainīgajos. Ar ielasītā faila datiem var tik veikta datu nosūtīšana uz citiem dokumentu izplatīšanas kanāliem, pārveidojot tos atbilstošājā formātā ar izejošo failu kartēšanu.
\par Izejošo failu kartēšana, kas ir izvadfaila izveide, ievietojot ielasītos datus faila formāta elementos, tiek izveidota atbilstoši klienta vēlamajam faila formātam un izkārtojumam. Klients atsūta testa failu kā piemēru rezultātam. Izstrādātājs izveido izejošo failu kartēšanu, atlasot no sistēmā esošajām dokumenta klasēm nepieciešamo informāciju un ievietojot to atbilstošajos tagos vai \texttt{csv} ailēs izvaddatos.
\par Ar ielasīto informāciju ir iespējams izveidot \texttt{pdf} formāta failu, pielietojot \texttt{xsl} stila atribūtus un elementus un ievietojot nepieciešamo informāciju atbilstošajos dokumenta apgabalos. \texttt{Pdf} formāta dokuments tiek izmantots informācijas vizuālai reprezentācijai pakalpojuma devēja klientam, izsūtot to pa pastu vai e-pastu, vai ar citiem saziņas un dokumentu nosūtīšanas kanāliem.
\par Izstrādātājs pēc nepieciešamības izgatavo e-pasta šablonu, kas līdzinās \texttt{pdf} veidnes izgatavošanai. Šablons tiek veidots, ievērojot \texttt{html} formātu teksta formatējumam un vizuālo elementu ģenerēšanai. Pakalpojuma devēja klients saņems dokumentu uz e-pastu jau izklāstītā formā.



% \begin{itemize}
    %     \item kā tehnoloģijas tiek sasaistītas
    %     \item demo
    %     \item mapping
    %     \item stylesheet
    %     \item test
    %     \item email
    % \end{itemize}