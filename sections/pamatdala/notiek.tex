\subsection{Kas tur notiek}
\par Uzņēmumam ir izstrādāta sistēma ar dažādām klasēm, kas visvairāk atbilst saņemto datu vērtībām. Visbiežāk saņemtie dati satur tādas vērtības, kā dokumenta pamatdetaļas( izveides un izpildes datumi, dokumenta numurs), parādu uzņēmuma pamatinformācija, parādnieka pamatinformācija, parāda detaļas(summas, preces vai pakalpojuma apraksts) dokumenta kopsummas un papildinformācija. Atbilstoši saņemtajiem datiem sistēmā ir izveidotas klases priekš juridiskajām un fiziskajām personām, dokumenta pamatinformācijai, dokumenta detaļām un dokumenta kopsavilkumam.
\par Klients nosūta testa failus, kas pēc iespējas vairāk atbilst reāli izmantotajiem failiem dokumentu izplatīšanā. Saņemot klienta testa failus izstrādātājs izveido ienākošo failu kartēšanu, ielasot faila sastāvu un saglabājot noteiktajās vietās esošās vērtības atbilstošajās klasēs un klašu mainīgajos. Ar ielasītā faila datiem var tik veikta datu nosūtīšana uz citiem dokumentu izplatīšanas kanāliem, pārveidojot tos atbilstošājā formātā ar izejošo failu kartēšanu.
\par Izejošo failu kartēšana tiek izveidota atbilstoši klienta vēlamajam failam. Klients atsūta testa failu kā piemēru rezultātam. Izstrādātājs izveido izejošo failu kartēšanu, atlasot no sistēmā esošajām dokumenta klasēm nepieciešamo informāciju un ievietojot to atbilstošajos tagos vai csv ailēs izvaddatos.
\par Ar ielasīto informāciju ir iespējams izveidot pdf formāta failu, pielietojot xsl stila atribūtus un elementus un ievietojot nepieciešamo informāciju atbilstošajos dokumenta apgabalos. Pdf formāta dokuments tiek izmantots informācijas vizālai reprezentācijai klienta klientam, izsūtot to ar pastu vai e-pastu, vai ar citiem saziņas un dokumentu nosūtīšanas kanāliem.
\par Izstrādātājs pēc nepieciešamības izgatavo e-pasta šablonu, kas līdzinās pdf veidnes izgatavošanai. Klienta klients saņems dokumentu uz epastu jau izklāstītā formā.



% \begin{itemize}
    %     \item kā tehnoloģijas tiek sasaistītas
    %     \item demo
    %     \item mapping
    %     \item stylesheet
    %     \item test
    %     \item email
    % \end{itemize}