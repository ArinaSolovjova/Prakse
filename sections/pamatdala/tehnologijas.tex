\subsection{Izmantotās tehnoloģijas}
\par Klases ir veidotas programmēšanas valodā \textit{Java}.
\par Kā programmēsanas koda rediģēšanas aplikācija tiek izmantota \textit{Eclipse IDE}.
\par \textbf{Apache Groovy} ir objektorientēta programmēšanas valoda \textit{Java} platformai.
\par \textbf{Smooks} ir programmēšanas valodas Java ietvars XML, CSV, EDI, Java un citu datu tipu apstrādei. Ar šo ietvaru tiek veikta ievaddatu kartēšana atbilstošajās klasēs un to mainīgajos. \textit{Smooks} pēc noklusējuma apstrādā XML datus, kur klašu instances tiek izveidoti uz noteiktiem tagiem un to mainīgo vērtības tiek ielasītas uz apakštagiem.

\par XML datu piemērs
{
    \setstretch{1.0}
    \begin{minted}[]{html}
    <invoice invoice-date="2020-01-15">
        <invoice-reference-number>12345</invoice-reference-number
        <invoice-due-date>2020-02-15</invoice-due-date>
    </invoice>
\end{minted}
}
\par Java klases piemērs
{
    \setstretch{1.0}
    \begin{minted}[]{java}
    public class Invoice {
        private Date creationDate;
        private Date dueDate;
        private Long referenceId;
        private String extraInfo;
        // ...
}
\end{minted}
}
\par Smooks kartēšanas kods
{
\setstretch{1.0}
\begin{minted}[]{html}
    <jb:bean beanId="invoice"
        class="com.baeldung.smooks.model.Invoice" createOnElement="invoice">
        <jb:value property="referenceId" data="invoice/invoice-reference-number" />
        <jb:value property="creationDate"
            data="invoice/@invoice-date" decoder="Date">
            <jb:decodeParam name="format">yyyy-MM-dd</jb:decodeParam>
        </jb:value>
        <jb:value property="dueDate"
            data="invoice/dueDate decoder="Date">
            <jb:decodeParam name="format">yyyy-MM-dd</jb:decodeParam>
        </jb:value>
    </jb:bean>
\end{minted}
}
%     \item izmantotās tehnoloģijas
%     \begin{itemize}
%         \item smooks
%         \item apache groovy
%         \item xsl
%         \item xml
%         \item txt
%         \item csv
%     \end{itemize}